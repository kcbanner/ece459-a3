\documentclass[12pt]{article}

\usepackage[letterpaper, hmargin=0.75in, vmargin=0.75in]{geometry}
\usepackage{float}
\usepackage{listings}

\pagestyle{empty}

\title{ECE 459: Programming for Performance\\Assignment 3}
\author{Casey Banner and Stephan van den Heuvel}
\date{March 25, 2013}

\lstset{
  frame=single,
  basicstyle=\ttfamily\footnotesize,
  framesep=10pt,
  aboveskip=10pt,
  belowskip=10pt
}


\begin{document}

\maketitle

\section*{Baseline Performance}

% - Talk about 25s avg execution time
% - Talk about Google Profiler, and how it is instrumented

% 25.641 + 25.348 + 26.820 + 26.817 + 25.530

The execution time of the provided assignment code over an average of
5 runs was $26.031$s. In order to understand where time was being spent
within the model the gperftools CPU profiler was used. For profiling
purposes, the test harness was instrumented as follows:

\begin{lstlisting}[language=C]
ProfilerStart("morph.profile");
m.morph(0, 0.5, 1.25, 0.2);
m.morph(1, 0.5, 1.25, 0.2);
ProfilerStop();
\end{lstlisting}

The resultant profile was then analyzed with the \texttt{pprof}
utility in order to get a line-by-line breakdown of how the execution
time was distributed. An excerpt of the output follows:

\begin{lstlisting}
     1      1  110:  double ww[lines];
    29     29  111:  QPoint pp[lines];
     .      .  112:
     .      .  113:  // for each line
     8      8  114:  for(int k=0; k<lines; ++k) {
     .      .  115:
     .      .  116:      // get original lines from reference line
    47     47  117:      QPoint P = listLines[h]->at(k).first;
     .      .  118:      QPoint Q = listLines[h]->at(k).second;
     .      .  119:
    35     35  120:      QVector2D XP(X - P);
    37     37  121:      QVector2D QP(Q - P);
     .      .  122:
    40     40  123:      QVector2D pQP(QP.y(), -QP.x());
     .      .  124:
     .      .  125:      // Calculate u, v
   414    414  126:      u = QVector2D::dotProduct(XP, QP) /  QP.lengthSquared();
     .      .  127:      v = QVector2D::dotProduct(XP, pQP) / QP.length();
     .      .  128:
     .      .  129:      // get interpolating lines from reference line
     7      7  130:      QPoint P2 = listAux[h]->at(k).first;
     .      .  131:      QPoint Q2 = listAux[h]->at(k).second;
     .      .  132:
    23     23  133:      QVector2D Q2P2(Q2 - P2);
    24     24  134:      QVector2D pQ2P2(Q2P2.y(), -Q2P2.x());
     .      .  135:
   438    438  136:      QVector2D X2 = QVector2D(P2) + u*Q2P2 + (v*pQ2P2)
                                        / Q2P2.length();
     .      .  137:
   126    126  138:      QPoint p = X2.toPoint() - X;
     .      .  139:

                         ...

     .      .  145:      double w = 0;
     1      1  146:      w =  pow(QP.length(), VARP);
     8      8  147:      w /= (VARA + dist);
    21     21  148:      w = pow(w, VARB);
     .      .  149:
     8      8  150:      ww[k] = w;
    11     11  151:      pp[k] = p;
     .      .  152:  }
     .      .  153:
     .      .  154:  QPoint sum(0.0, 0.0);
     .      .  155:  double wsum = 0;
    45     45  156:  for(int k=0; k<lines; ++k) {
   261    261  157:      sum  += ww[k] * pp[k];
     .      .  158:      wsum += ww[k];
     .      .  159:  }
   108    108  160:  sum /= wsum;
\end{lstlisting}

From this analysis we can see that the areas where the most time is spent are:

% - length() is slow
% - big calculations are slow (u, v)
% - the final summation loop is slow

\begin{itemize}
  \item The \texttt{u} and \texttt{v} calcuations
  \item The \texttt{X2} calculation
  \item The weighted sum calcuation
  \item The calculation of \texttt{w} and \texttt{p}
\end{itemize}

The \texttt{w} calculation involves expensive calls to \texttt{pow}
with fractional exponents, and we can see the effects of this in the
\texttt{pprof --text} output:

\begin{lstlisting}
  2365  37.0%  37.0%     2365  37.0% __log10_finite
  1784  27.9%  64.8%     1784  27.9% Model::morph
  1381  21.6%  86.4%     1381  21.6% __exp_finite
   230   3.6%  90.0%      230   3.6% 00007f19fa8f8b7f
   120   1.9%  91.9%      120   1.9% pow
    61   1.0%  92.9%       61   1.0% 00007f19fa34ecec
\end{lstlisting}

\section*{Improvements}

\subsection*{Precalculation and SIMD}

The first changes that were done were improvements to the sequential
execution time. The change involved precalculating everything that was
only dependant on \texttt{k} a single time before the \texttt{for}
loops. All of the precalculated results were stored in arrays that
were keyed on \texttt{k}. These arrays were put on the stack in order
to improve locality. Also, an optimized version of the weighted sum
loop was created using GCC's SIMD intrinsics for the special case of
\texttt{lines = 4}. 

The code diff is as follows:

\begin{lstlisting}[caption={Precalculation and SIMD code diff}]
diff --git a/model.cpp b/model.cpp
index 614f25b..23fdea9 100644
--- a/model.cpp
+++ b/model.cpp
@@ -1,2 +1,9 @@
 #include "model.h"
 
+typedef double  v4df  __attribute__ ((vector_size (32)));  /* double[4], AVX  */
+
+union vec4d {
+	v4df v;
+	double a[4];
+};
+

 Model::Model() {
-    for (unsigned int i = 0; i < NUM_IMAGES; ++i) imgs[i] = new QImage();
+  for (unsigned int i = 0; i < NUM_IMAGES; ++i) imgs[i] = new QImage();
 
-    listLines = new vector<pair <QPoint, QPoint> >* [2];
-    listAux = new vector<pair <QPoint, QPoint> >* [2];
+  listLines = new vector<pair <QPoint, QPoint> >* [2];
+  listAux = new vector<pair <QPoint, QPoint> >* [2];
 }
  
 void Model::morph(int h, double VARA, double VARB, double VARP) {
-    int n = 0;
-    int lines = listLines[0]->size();
-
-    for(int i=0; i<wimg; ++i) {
-        for(int j=0; j<himg; ++j) {
-            
-            QPoint X(i, j);
-            double u, v;
-            
-            double ww[lines];
-            QPoint pp[lines];
+  int lines = listLines[0]->size();
+  
+  pair<QPoint, QPoint>* linesData = listLines[h]->data();
+  pair<QPoint, QPoint>* auxData = listAux[h]->data();
+
+  // Memoize  
+
+  double QPs_x[lines];
+  double QPs_y[lines];  
+  double pQPs_x[lines];
+  double pQPs_y[lines];
+
+  QVector2D Q2P2s[lines];
+  QVector2D pQ2P2s[lines];
+  double QPlengths[lines];
+  double QPlengthsSquared[lines];
+  double Q2P2lengths[lines];
+  double powVARP[lines];
+  for(int k = 0; k < lines; ++k) {
+    QVector2D QP;
+    QVector2D pQP;
+    
+    QPoint P = linesData[k].first;
+    QPoint Q = linesData[k].second;
+    QPoint P2 = auxData[k].first;
+    QPoint Q2 = auxData[k].second;
+
+    QP = QVector2D(Q - P);
+    pQP = QVector2D(QP.y(), -QP.x());
+    QPlengthsSquared[k] = QP.lengthSquared();
+    QPlengths[k] = QP.length();
+    
+    QPs_x[k] = QP.x();
+    QPs_y[k] = QP.y();
+    pQPs_x[k] = pQP.x();
+    pQPs_y[k] = pQP.y();
+
+    powVARP[k] = pow(QPlengths[k], VARP);
+    Q2P2s[k] = QVector2D(Q2 - P2);
+    pQ2P2s[k] = QVector2D(Q2P2s[k].y(), -Q2P2s[k].x());
+    Q2P2lengths[k] = Q2P2s[k].length();
+  }
+
+  for(int i=0; i<wimg; ++i) {
+    for(int j=0; j<himg; ++j) {
+      QPoint X(i, j);
             
-            // for each line
-            for(int k=0; k<lines; ++k) {
-                
-                // get original lines from reference line
-                QPoint P = listLines[h]->at(k).first;
-                QPoint Q = listLines[h]->at(k).second;
-                
-                QVector2D XP(X - P);
-                QVector2D QP(Q - P);
-                
-                QVector2D pQP(QP.y(), -QP.x());
-                
-                // Calculate u, v
-                u = QVector2D::dotProduct(XP, QP) /  QP.lengthSquared();
-                v = QVector2D::dotProduct(XP, pQP) / QP.length();
-
-                // get interpolating lines from reference line
-                QPoint P2 = listAux[h]->at(k).first;
-                QPoint Q2 = listAux[h]->at(k).second;
-
-                QVector2D Q2P2(Q2 - P2);
-                QVector2D pQ2P2(Q2P2.y(), -Q2P2.x());
-
-                QVector2D X2 = QVector2D(P2) + u * Q2P2 + (v * pQ2P2) /
                                Q2P2.length();
-
-                QPoint p = X2.toPoint() - X;
-
-                double dist = 0;
-                if(u > 0 && u < 1) dist = fabs(v);
-                else if(u <= 0) dist = sqrt(pow(X.x() - P.x(), 2.0) + 
                                             pow(X.y() - P.y(), 2.0));
-                else dist = sqrt(pow(X.x() - Q.x(), 2.0) +
                                  pow(X.y() - Q.y(), 2.0));
-
-                double w = 0;
-                w =  pow(QP.length(), VARP);
-                w /= (VARA + dist);
-                w = pow(w, VARB);
-
-                ww[k] = w;
-                pp[k] = p;
-            }
-
-            QPoint sum(0.0, 0.0);
-            double wsum = 0;
-            for(int k=0; k<lines; ++k) {
-                sum  += ww[k] * pp[k];
-                wsum += ww[k];
-            }
-            sum /= wsum;
-
-            QPoint X2 = X + sum;
-
-            double y0, x0;
-            x0 = ceil(X2.x());
-            if(x0 < 0) x0 = 0;
-            if(x0 >= wimg) x0 = wimg-1;
-
-            y0 = ceil(X2.y());
-            if(y0 < 0) y0 = 0;
-            if(y0 >= himg) y0 = himg-1;
-
-            X2 = QPoint(x0, y0);
-
-            if(X2 == X) n++;
-            imgs[h+2]->setPixel(X, imgs[h]->pixel(X2));
+      vec4d ww;
+      vec4d pp_x;
+      vec4d pp_y;
+      
+      double u[lines];
+      double v[lines];
+
+      for(int k = 0; k < lines; ++k) {
+        QPoint P = linesData[k].first;
+        double XPx = X.x() - P.x();
+        double XPy = X.y() - P.y();
+
+        u[k] = (XPx * QPs_x[k] + XPy * QPs_y[k]) / QPlengthsSquared[k];
+        v[k] = (XPx * pQPs_x[k] + XPy * pQPs_y[k]) / QPlengths[k];        
+      }
+
+      QVector2D X2s[lines];
+      for(int k = 0; k < lines; ++k) {
+        QPoint P2 = auxData[k].first;
+        X2s[k] = QVector2D(P2) + u[k] * Q2P2s[k] + (v[k] * pQ2P2s[k]) /
                            Q2P2lengths[k];
+      }
+
+      for(int k = 0; k < lines; ++k) {
+        QPoint P = linesData[k].first;
+
+        double dist = 0;
+        if(u[k] > 0 && u[k] < 1) {
+          dist = fabs(v[k]);
+        } else {
+          QPoint Q = linesData[k].second;
+
+          if(u[k] <= 0) {
+            dist = sqrt(pow(X.x() - P.x(), 2.0) + pow(X.y() - P.y(), 2.0));
+          } else {
+            dist = sqrt(pow(X.x() - Q.x(), 2.0) + pow(X.y() - Q.y(), 2.0));
+          }
+        }
+
+        double w;
+        w =  powVARP[k];
+        w /= (VARA + dist);
+        w = pow(w, VARB);
+
+        ww.a[k] = w;
+        pp_x.a[k] = qRound(X2s[k].x()) - X.x();
+        pp_y.a[k] = qRound(X2s[k].y()) - X.y();
+      }
+
+      int sum_x = 0;
+      int sum_y = 0;
+      double wsum = 0.0;
+	    
+      if (lines == 4) {
+        vec4d product;
+        vec4d sum;
+
+        product.v = __builtin_ia32_mulpd256(ww.v, pp_x.v);
+        sum.v = __builtin_ia32_roundpd256(product.v, 0);
+        sum_x = (int)sum.a[0] + (int)sum.a[1] + (int)sum.a[2] + (int)sum.a[3];
+	
+        product.v = __builtin_ia32_mulpd256(ww.v, pp_y.v);
+        sum.v = __builtin_ia32_roundpd256(product.v, 0);
+        sum_y = (int)sum.a[0] + (int)sum.a[1] + (int)sum.a[2] + (int)sum.a[3];
+
+        wsum += ww.a[0];
+        wsum += ww.a[1];
+        wsum += ww.a[2];
+        wsum += ww.a[3];
+      } else {
+        printf("error\n");
+
+        for(int k=0; k<lines; ++k) {
+          sum_x  += ww.a[k] * pp_x.a[k];
+          sum_y  += ww.a[k] * pp_y.a[k];			
+          wsum += ww.a[k];
         }
+      }
+
+      sum_x = qRound(sum_x / wsum);
+      sum_y = qRound(sum_y / wsum);
+
+      QPoint X2 = X + QPoint(sum_x, sum_y);
+
+      double y0, x0;
+      x0 = ceil(X2.x());
+      if(x0 < 0) x0 = 0;
+      if(x0 >= wimg) x0 = wimg-1;
+
+      y0 = ceil(X2.y());
+      if(y0 < 0) y0 = 0;
+      if(y0 >= himg) y0 = himg-1;
+
+      X2 = QPoint(x0, y0);
+
+      imgs[h+2]->setPixel(X, imgs[h]->pixel(X2));
     }
+  }
 } 
\end{lstlisting}

This code was run five times again to measure the improved execution
time. The average runtime after these improvements were made was
$17.5882$s. The results of the benchmark can be seen in
Table~\ref{tbl-precalculation}.

% >>> (17.783 + 17.635 + 17.447 + 17.249 + 17.827) / 5
% 17.5882

\begin{table}[H]
  \centering
  \begin{tabular}{lr}
    & {\bf Time (s)} \\
    \hline
    Run 1 & 17.783 \\
    Run 2 & 17.635 \\
    Run 3 & 17.447 \\
    Run 4 & 17.249 \\
    Run 5 & 17.827 \\
    \hline
    Average & 17.5882 \\
  \end{tabular}
  \caption{Benchmark results for precalculation and SIMD optimizations}
  \label{tbl-precalculation}
\end{table}

This optimization improved runtime because instead of doing the
calculations for \texttt{QP}, \texttt{pQP}, \texttt{Q2P2},
\texttt{pQ2P2}, the lengths of \texttt{QP} and \texttt{Q2P2}, and the
\texttt{pow} call every pixel, they are done a single time. The cost
of these optimizations is increased memory usage, but that usage is
far eclipsed by the memory required to store the resultant image.

The SIMD optimization was beneficial because it allowed all the
multiplcations between \texttt{ww} and \texttt{pp} to be performed in
two instructions. The expensive \texttt{qRound} calls that were
happening in \texttt{QPoint::operator*=} were also optimized away into
an AVX instruction that does rounding.

\subsection*{Threading}

\end{document}
