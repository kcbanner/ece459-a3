\documentclass[12pt]{article}

\usepackage[letterpaper, hmargin=0.75in, vmargin=0.75in]{geometry}
\usepackage{float}
\usepackage{listings}

\pagestyle{empty}

\title{ECE 459: Programming for Performance\\Assignment 3}
\author{Casey Banner and Stephan van den Heuvel}
\date{March 25, 2013}

\lstset{
  frame=single,
  basicstyle=\ttfamily\footnotesize,
  framesep=10pt,
  aboveskip=10pt,
  belowskip=10pt
}


\begin{document}

\maketitle

\section*{Baseline Performance}

% - Talk about 25s avg execution time
% - Talk about Google Profiler, and how it is instrumented

The execution time of the provided assignment code over an average of
5 runs was $26.031$s. In order to understand where time was being spent
within the model the gperftools CPU profiler was used. For profiling
purposes, the test harness was instrumented as follows:

\begin{lstlisting}[language=C]
ProfilerStart("morph.profile");
m.morph(0, 0.5, 1.25, 0.2);
m.morph(1, 0.5, 1.25, 0.2);
ProfilerStop();
\end{lstlisting}

The resultant profile was then analyzed with the \texttt{pprof}
utility in order to get a line-by-line breakdown of how the execution
time was distributed. An excerpt of the output follows:

\begin{lstlisting}
     1      1  110:  double ww[lines];
    29     29  111:  QPoint pp[lines];
     .      .  112:
     .      .  113:  // for each line
     8      8  114:  for(int k=0; k<lines; ++k) {
     .      .  115:
     .      .  116:      // get original lines from reference line
    47     47  117:      QPoint P = listLines[h]->at(k).first;
     .      .  118:      QPoint Q = listLines[h]->at(k).second;
     .      .  119:
    35     35  120:      QVector2D XP(X - P);
    37     37  121:      QVector2D QP(Q - P);
     .      .  122:
    40     40  123:      QVector2D pQP(QP.y(), -QP.x());
     .      .  124:
     .      .  125:      // Calculate u, v
   414    414  126:      u = QVector2D::dotProduct(XP, QP) /  QP.lengthSquared();
     .      .  127:      v = QVector2D::dotProduct(XP, pQP) / QP.length();
     .      .  128:
     .      .  129:      // get interpolating lines from reference line
     7      7  130:      QPoint P2 = listAux[h]->at(k).first;
     .      .  131:      QPoint Q2 = listAux[h]->at(k).second;
     .      .  132:
    23     23  133:      QVector2D Q2P2(Q2 - P2);
    24     24  134:      QVector2D pQ2P2(Q2P2.y(), -Q2P2.x());
     .      .  135:
   438    438  136:      QVector2D X2 = QVector2D(P2) + u*Q2P2 + (v*pQ2P2)
                                        / Q2P2.length();
     .      .  137:
   126    126  138:      QPoint p = X2.toPoint() - X;
     .      .  139:

                         ...

     .      .  145:      double w = 0;
     1      1  146:      w =  pow(QP.length(), VARP);
     8      8  147:      w /= (VARA + dist);
    21     21  148:      w = pow(w, VARB);
     .      .  149:
     8      8  150:      ww[k] = w;
    11     11  151:      pp[k] = p;
     .      .  152:  }
     .      .  153:
     .      .  154:  QPoint sum(0.0, 0.0);
     .      .  155:  double wsum = 0;
    45     45  156:  for(int k=0; k<lines; ++k) {
   261    261  157:      sum  += ww[k] * pp[k];
     .      .  158:      wsum += ww[k];
     .      .  159:  }
   108    108  160:  sum /= wsum;
\end{lstlisting}

From this analysis we can see that the areas where the most time is spent are:

% - length() is slow
% - big calculations are slow (u, v)
% - the final summation loop is slow

\begin{itemize}
  \item The \texttt{u} and \texttt{v} calcuations
  \item The \texttt{X2} calculation
  \item The weighted sum calcuation
  \item The calculation of \texttt{w} and \texttt{p}
\end{itemize}

The \texttt{w} calculation involves expensive calls to \texttt{pow}
with fractional exponents, and we can see the effects of this in the
\texttt{pprof --text} output:

\begin{lstlisting}
  2365  37.0%  37.0%     2365  37.0% __log10_finite
  1784  27.9%  64.8%     1784  27.9% Model::morph
  1381  21.6%  86.4%     1381  21.6% __exp_finite
   230   3.6%  90.0%      230   3.6% 00007f19fa8f8b7f
   120   1.9%  91.9%      120   1.9% pow
    61   1.0%  92.9%       61   1.0% 00007f19fa34ecec
\end{lstlisting}

\section*{Improvements}

\subsection*{Memoization and SIMD}

\subsection*{Threading}

\end{document}
