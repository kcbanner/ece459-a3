
\documentclass[12pt]{article}

\usepackage[letterpaper, hmargin=0.75in, vmargin=0.75in]{geometry}
\usepackage{float}
\usepackage{listings}

\pagestyle{empty}

\title{ECE 459: Programming for Performance\\Assignment 4}
\author{Casey Banner and Stephan van den Heuvel}
\date{April 8, 2013}

% Code listing style
\lstset{frame=single}

\begin{document}

\maketitle

\section{Design Decisions}
To produce the optimized OpenCL version we followed two basic ideas. One was that every 
step of the algorithm
that depended on a previous step would be a new kernel. This provided the easiest form of 
synchronization on
these dependencies. Therefore initially we had three kernels. One to compute bins, one to 
sort the bins, and 
finally one to do the actual body-body interaction.

The second goal was the minimize data transmission between the GPU and CPU. We 
benchmarked this quickly and
found it to be comparatively slow. In light of this, we changed a computation that was on 
the CPU, computing
offsets for the bin index array, to run on the GPU. This added one more kernel to our 
program. This let the associated data stay in GPU memory and not require a transfer to 
CPU memory.

\section{Results}

The performance of the naive OpenCL implementation is given in
Table~\ref{tbl-naive-bench} and the optimized version's benchmark is
provided by Table~\ref{tbl-opt-bench}.

\begin{table}[H]
  \centering
  \begin{tabular}{lr}
    & {\bf Time (s)} \\
    \hline
    Run 1 & 0.176 \\
    Run 2 & 0.170 \\
    Run 3 & 0.164 \\
    Run 4 & 0.172 \\
    Run 5 & 0.169 \\
    \hline
    Average & 0.170 \\
  \end{tabular}
  \caption{Benchmark results for naive OpenCL Implementation}
  \label{tbl-naive-bench}
\end{table}

\begin{table}[H]
  \centering
  \begin{tabular}{lr}
    & {\bf Time (s)} \\
    \hline
    Run 1 & 0.207 \\
    Run 2 & 0.199 \\
    Run 3 & 0.208 \\
    Run 4 & 0.201 \\
    Run 5 & 0.197 \\
    \hline
    Average & 0.202 \\
  \end{tabular}
  \caption{Benchmark results for optimized OpenCL Implementation}
  \label{tbl-opt-bench}
\end{table}

The naive implementation far out performs the sequential CPU-only
version, which runs around 15 seconds on ECE459-1. The naive
implementation also slightly outperforms the optimized version for
32000 points. The optimization comes with overhead. In addition for
this number of points, memory transfer to and from the GPU is most
likely the bottle-neck. For larger number of points, the optimization
may become viable.

\end{document}

